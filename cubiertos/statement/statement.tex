\documentclass{oci}
\usepackage[utf8]{inputenc}
\usepackage{lipsum}

\title{Cubiertos}
\codename{cubiertos}

\begin{document}
\begin{problemDescription}
Todos los días, el padre de Nelmancito le cocina y le envía al colegio un potecito con comida junto con un cuchillo y un tenedor. Desgraciadamente, nuestro amigo Nelmancito es un poco descuidado y suele perder uno o ambos cubiertos.

Su padre detesta tener una cantidad diferente de cuchillos y tenedores, por lo que decide invertir dinero en cubiertos hasta lograr tener la misma cantidad de ambos. Para lograr su cometido tiene dos opciones: o bien comprar nuevos cubiertos, o bien reciclar algunos de sus cubiertos.

Por ejemplo, si el papá de Nelmancito tiene 7 tenedores y 12 cuchillos hay muchas formas en las cuales puede lograr su objetivo. Por ejemplo, las siguientes son tres posibles formas: 
\begin{itemize}
	\item comprar 5 tenedores (se quedaría con 12 tenedores y 12 cuchillos),
	\item comprar 3 tenedores y reciclar 2 cuchillos (se quedaría con 10 tenedores y 10 cuchillos),
	\item reciclar 2 tenedores y reciclar 7 cuchillos (se quedaría con 5 tenedores y 5 cuchillos).
\end{itemize}

Ahora bien, comprar un tipo de cubierto tiene un costo asociado, del mismo modo que reciclar también tiene un costo. 
Dada la cantidad de cuchillos y tenedores que tiene el padre de Nelmancito y los precios de comprar y reciclar cada cubierto, tu tarea
es ayudarlo a conseguir su objetivo gastando la menor cantidad de dinero posible.


\end{problemDescription}

\begin{inputDescription}
Cada input consiste en dos líneas. La primera línea contiene tres enteros, $T$
$P_t$ y $R_t$ separados por un espacio, que codifican respectivamente la
cantidad de tenedores, el precio de comprar un tenedor y el precio de reciclar
un tenedor. La segunda línea contiene tres enteros, $C$ $P_c$ y
$R_c$, que codifican la cantidad de cuchillos, el precio de comprar un cuchillo
y el precio de reciclar un cuchillo.
Todos los enteros en el input serán números entre $0$ y $1000$ ($0\leq T,C,P_t,P_c,R_t,R_c, \leq 1000$).
\end{inputDescription}

\begin{outputDescription}
Debes entregar un entero describiendo el mínimo costo que deberá pagar el padre de Nelmancito para cumplir su objetivo.
\end{outputDescription}

\newpage

\begin{scoreDescription}
\score{20} $C=T+1$ y  $R_t= R_c = P_t = P_c$ (o sea, hay exactamente un cuchillo más que un tenedor y el precio de comprar y reciclar es el mismo para todos los cubiertos).
\score{20} $C=T+1$ (o sea, hay exactamente un cuchillo más que un tenedor, pero los precios de reciclar o comprar pueden ser distintos)
\score{20} $R_t= R_c = P_t = P_c$ (o sea, el precio de reciclar y comprar es el mismo para cuchillos y tenedores, pero puede haber una cantidad arbitraria de cuchillos y tenedores)
\score{40} Sin restricciones adicionales.
%  \score{20} $1 \leq T,C \leq 1000$, $C=T+1$ (o sea hay exactamente un cuchillo más que un tenedor) $ 0 \leq P_t, P_c, R_t, R_c, \leq 10$.
%  \score{20} $1 \leq T,C \leq 1000$, $ 0 \leq R_t= R_c = P_t = P_c \leq 10^3$.
%  \score{60} $1 \leq T,C \leq 1000$, $ 0 \leq P_t, P_c, R_t, R_c, \leq 10^3$.
\end{scoreDescription}

\begin{sampleDescription}
\sampleIO{sample-1}
\sampleIO{sample-2}
\end{sampleDescription}

\end{document}
