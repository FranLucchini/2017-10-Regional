\documentclass{oci}
\usepackage[utf8]{inputenc}
\usepackage{lipsum}

\title{Chun-li}
\codename{chunli}

\begin{document}
\begin{problemDescription}
Chun-li es un personaje icónico de la saga Street Fighters, y kikou-shou es una de sus habilidades especiales más poderosas, en las que concentra todo su poder para crear una bola de energía que daña a todo a su alrededor.

En el nuevo videojuego Ocis Fighters, hay una modalidad de juego en donde Chun-li debe enfrentarse a una orda de zombies, los zombies son muy débiles y fáciles de matar, sin embargo el peligro está en la cantidad de zombies que la atacan.

Para pasar de nivel, Chun-li debe eliminar a todos los zombies del escenario.

Chun-li siempre está en la coordenada (0, 0).

Dado un input con los puntos en el plano en donde están zombies, se requiere un programa que entregue el radio mínimo R del Kikou-shou que debe ejecutar Chun-li para matar a todos los zombies.

Un zombie es asesinado por el Kikou-shou de Chun-li, si la distancia entre la coordenada del zombie (x,y) y la posición de Chun-li (0, 0) es menor o igual que el radio R del Kikou-shou.

\end{problemDescription}

\begin{inputDescription}
La primera linea contiene un número N, la cantidad de zombies.

A continuación se entregan N lineas con dos números separados por espacio $X_i$ y $Y_i$ que representan a las coordenadas del i-ésimo zombie, $X_i$ y $Y_i$ nunca son ambos 0 para algún i.

\end{inputDescription}

\begin{outputDescription}

Imprimir una linea con un único numero R señalando el mínimo radio entero para matar a todos los zombies con un único Kikou-shou.

\end{outputDescription}

\begin{scoreDescription}
\score{nlehmann define los puntajes} $N \leq 10^2, \left | X_i \right | \leq 10^3,  \left | Y_i \right | \leq 10^3$.
\score{nlehmann define los puntajes} $N \leq 10^5, \left | X_i \right | \leq 10^4,  Y_i = 0$.
\score{nlehmann define los puntajes} $N \leq 10^3, \left | X_i \right | \leq 10^4,  \left | Y_i \right | \leq 10^4$.
\score{nlehmann define los puntajes} $N \leq 10^5, \left | X_i \right | \leq 10^9,  \left | Y_i \right | \leq 10^9$.
\end{scoreDescription}

\begin{sampleDescription}
\sampleIO{sample-1}
\sampleIO{sample-2}
\end{sampleDescription}

\end{document}
