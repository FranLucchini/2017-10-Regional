\documentclass{oci}
\usepackage[utf8]{inputenc}

\title{Ocimatic}
\codename{ocimatic}

\begin{document}
\begin{problemDescription}
  Hoy en día el correo electrónico es sin duda una de las herramientas más
  utilizadas.
  Los usuarios más asiduos pueden enviar decenas o incluso cientos de correos
  diarios.
  El alto volumen de correos es un problema para estos usuarios, quienes deben
  recordar cuáles no han recibido respuesta y así, en caso de ser necesario,
  intentar comunicarse nuevamente con el receptor del correo.
  Ocimatic es una herramienta que ayuda a lidiar con este problema, enviando
  notificaciones a sus usuarios cada vez que alguno de sus correos no ha
  recibido respuesta dentro de un plazo determinado.
  
  Ocimatic funciona de forma muy sencilla.
  Los usuarios puede configurar el plazo $D$ en días que están dispuestos a
  esperar por una respuesta.
  Cada vez que un correo es enviado, Ocimatic configura un evento para que se
  ejecute luego de $D$ días.
  Cuando el evento es ejecutado transcurridos los $D$ días, Ocimatic verifica
  si el correo ya recibió una respuesta y notifica al usuario en caso de no
  haber recibido una.
  Por ejemplo, si un usuario tiene configurado el plazo en 2 días y envía un
  correo un lunes, Ocimatic deberá verificar el miércoles si el correo ya recibió
  una respuesta.

  Muchos de los usuarios han comenzado a quejarse porque desearían que Ocimatic
  solo considerara los días hábiles dentro del plazo.
  Por ejemplo, si un usuario tiene configurado el plazo en 2 días y envía un
  correo el viernes, no espera ser notificado el domingo, sino el martes.
  De la misma forma, si envía un correo el sábado o el domingo esperaría ser
  notificado el miércoles.

  Tomando en cuenta las quejas de sus usuarios los desarrolladores quieren
  modificar Ocimatic para que solo considere los días hábiles.
  Para esto, necesitan saber la cantidad total de días, considerando los fines
  de semana, que Ocimatic debe esperar dada la cantidad de días hábiles $D$ que
  el usuario configuró.
\end{problemDescription}

\begin{inputDescription}
  Un caso de prueba corresponde al envío de un correo y está descrito en una
  línea con dos enteros $D$ y $W$.
  $D$ corresponde a la cantidad de días hábiles que el usuario está dispuesto a
  esperar.
  $W$ es un entero entre 1 y 7 que corresponde al día de la semana en que el
  correo fue enviado (lunes=1, martes=2, miércoles=3, jueves=4, viernes=5,
  sábado=6 y domingo=7).
\end{inputDescription}

\begin{outputDescription}
  La salida debe contener un único entero correspondiente a la cantidad total de
  días que Ocimatic debe esperar antes de verificar si el correo recibió
  respuesta.
\end{outputDescription}

\begin{scoreDescription}
  \score{5} Se probarán varios casos donde $W=1$ y $0 < D \leq 4$.
  \score{10} Se probarán varios casos donde $1 \leq W \leq 5$ y $0 < D \leq 5$.
  \score{20} Se probarán varios casos donde $0 < D \leq 10$.
  \score{30} Se probarán varios casos donde $0 < D \leq 10^5$
  \score{35} Se probarán varios casos donde $0 < D \leq 10^{10}$
\end{scoreDescription}

\begin{sampleDescription}
\sampleIO{sample-1}
\sampleIO{sample-2}
\end{sampleDescription}

\end{document}
