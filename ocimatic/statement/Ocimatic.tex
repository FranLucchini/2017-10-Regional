\documentclass{oci}
\usepackage[utf8]{inputenc}

\title{Ocimatic}
\codename{ocimatic}

\begin{document}

\section*{Pauta}
\begin{scoreDescription}
  \score{5} Se probarán varios casos donde $W=1$ y $0 < D \leq 4$.

  Lo primero que debemos considerar, es que siempre es Lunes en esta subtarea.
  El máximo de días es 4, por lo que a lo más, se esperarán cuatro días. Eso
  significa que no tendremos que contar días del fin de semana. Por lo tanto, los
  días totales de espera son la misma cantidad que $D$.
  
  \vspace{10mm}
  \score{10} Se probarán varios casos donde $1 \leq W \leq 5$ y $0 < D \leq 5$.

  En esta subtarea, ya no sirve la soluci\'on anterior, porque hay m\'ultiples casos
  en los que se deben contar d\'ias del fin de semana para el total. Por ejemplo, si 
  $D = 4$ y $W = 2$ (Martes), el cuarto d\'ia es S\'abado y Ocimatic esperar\'a hasta
  el Lunes para mandar el aviso, por lo que son 6 d\'ias de espera.
  
  Luego, considerando los l\'imites de $D$ y $W$, s\'olo pasaremos por 1 fin 
  de semana en el peor caso. El caso m\'as largo ocurre cuando $D = 5$ y $W = 5$ (Viernes).
  Como el correo fue enviado el d\'ia $5$, la cuenta total pasa por el fin de semana
  y luego avanza cinco d\'ias hasta llegar al siguiente Viernes. Gracias a eso, sabemos
  que no llega a m\'as de un fin de semana.

  Desde ahora en adelante, una cosa importante que debemos hacer es disminuir $W$ en uno para 
  que la cuenta comience desde $0$ en vez de $1$.
  
  Entonces, tenemos dos opciones:
  \begin{enumerate}
    \item Si los valores de entrada no pasan por el fin de semana, retornamos $D$, al igual que en el primer caso.
    \item Si los valores de entrada pasan por el fin se semana, retornamos $D+2$, por que se deben contar dos d\'ias del fin de semana.
  \end{enumerate}

  Para saber si pasa por el fin de semana, la suma de $D + W$ debe superar el valor $4$.
  Esta suma supera $4$ siempre que la espera se pasa del Viernes.

  \vspace{10mm}
  \score{18} Se probarán varios casos donde $0 < D \leq 10$.

  Ahora, han aumentado los l\'imites de $D$ y adem\'as $W$ ya no tiene restricci\'on,
  o sea que puede tomar un valor de $1$ a $7$. La soluci\'on anterior ya no es suficiente,
  porque como $D$ puede llegar hasta $10$, podemos llegar a contar m\'as de un fin de 
  semana. Tambi\'en, debemos considerar qu\'e pasa si el correo se env\'ia un S\'abado
  o Domingo y lo debemos tratar como si se mandara el Lunes siguiente. 

  Al igual que antes, debemos disminuir $W$ en uno para que la cuenta comience desde $0$
  en vez de $1$.

  El primer caso es cuando $W$ es un d\'ia de semana ($0 \leq W \leq 4$): si la suma de
  $D + W$ supera $4$, debemos a\~nadir $2$ d\'ias extras, como en el 
  \'item anterior. Pero debemos asegurarnos de que no llegue a dos fines de semana.
  Este l\'imite ocurre cuando la espera de d\'ias $D$ supera una semana h\'abil completa.
  Como la semana h\'abil tiene $5$ d\'ias, el segundo Viernes corresponde al valor $9$.
  Ese es nuestro l\'imite superior, por lo que $W + D \leq 9$
  
  Ahora, si $D + W$ superan $9$, eso significa que ahora se deben contar $2$
  fines de semana al total de d\'ias de espera $D$, por lo que se a\~naden $4$ d\'ias a
  la suma.
  
  El segundo caso es cuando $W$ es S\'abado o Domingo: en este caso, no vamos a considerar
  $W$ para cu\'antos d\'ias de fin debemos a\~nadir. Esto lo hacemos as\'i porque enviar 
  un correo el fin de semana es equivalente a mandarlo el Lunes que sigue, o sea, $W = 0$.
  Pero, no debemos olvidar incluir estos d\'ias a la espera final.
  
  Entonces, si el correo se env\'ia un S\'abado, debemos incorporar a los d\'ias de espera totales
  tanto S\'abado como Domingo. Si el primer d\'ia era Domingo, debemos incorporar a la espera 
  s\'olo el Domingo.
  
  Despu\'es, hacemos algo lo mismo que antes: si $D + W$ supera $4$, debemos a\~nadir
  $2$ d\'ias extra, si supera $9$, debemos aumentar en $4$.

  \vspace{10mm}
  \score{28} Se probarán varios casos donde $0 < D \leq 10^5$

  Ahora que $D$ tiene l\'imites muy grandes, debemos pensar una soluci\'on que sea 
  general. Las que hemos hecho hasta el momento, s\'olo manejan casos particulares.
  De todas formas, podemos observar de ellas que hay un patr\'on que podemos seguir.

  Para cada soluci\'on usamos $D + W$ para saber el n\'umero de fines de semana
  que debemos a\~nadir. Para obtener ese n\'umero, debemos dividir $D + W$ por $5$,
  ya que esto nos indica c\'antas veces nos pasamos del viernes y eso sucede cada vez
  que esta suma llega a un m\'ultiplo de $5$, que representa los $5$ d\'ias de la semana 
  de trabajo.

  Luego, esta cantidad se multiplica por $2$ para obtener la cantidad de d\'ias que se
  deben a\~nadir a la espera.

  Teniendo el total de d\'ias de fin de semana que nos faltan, solo nos queda sumarlos a
  $D$, que son los d\'ias h\'abiles de espera.

  Al igual que antes, si se env\'ia un correo el fin de semana, debemos interpretarlo como
  si se mandara el Lunes que sigue (fijamos $W=0$). No debemos olvidar agreagar estos primeros
  d\'ias a la suma total de espera.

  \vspace{10mm}
  \score{39} Se probarán varios casos donde $0 < D \leq 10^{10}$

  Lo m\'as importante que debemos considerar en esta  \'ultima subtarea es que $D$ puede llegar a
  tener un valor de $10^{10}$. Hasta ahora hemos usados variables de tipo $int$, pero el valor m\'as
  grande que tiene este tipo de dato es $2,147,483,647$ o $4,294,967,295$ si fuera un $unsigned\ int$.
  
  Ambos valores son menores a $10^{10}$, por lo que debemos usar variables de tipo $long\ long$ que puede
  guardar valores mucho mayores, esto sirve tanto para $D$ como para guardar la suma total. No es necesario cambiar
  el tipo de dato de $W$.

  Tambi\'en, aprovechamos de agrupar los casos de cuando $W$ es un S\'abado o Domingo en un mismo bloque.
  Si restamos al n\'umero de d\'ias de la semana el valor de $W$, obtenemos cu\'antos d\'ias extra debemos sumar:
  \begin{enumerate}
    \item Cuando es S\'abado, o sea $W = 5$: $7 - W = 2$
    \item Cuando es Domingo, o sea $W = 6$: $7 - W = 1$
  \end{enumerate}
\end{scoreDescription}

\end{document}
