\documentclass{oci}
\usepackage[utf8]{inputenc}

\title{Ocimatic}
\codename{ocimatic}

\begin{document}
\begin{problemDescription}
  Hoy en día el correo electrónico es sin duda una de las herramientas más
  utilizadas.
  Los usuarios más asiduos pueden enviar decenas o incluso cientos de correos
  diarios.
  El alto volumen de correos es un problema para estos usuarios, quienes deben
  recordar cuáles no han recibido respuesta y así, en caso de ser necesario,
  intentar comunicarse nuevamente con el receptor del correo.
  \textbf{Ocimatic es una herramienta que ayuda a lidiar con este problema, enviando
  notificaciones a sus usuarios cada vez que alguno de sus correos no ha
  recibido respuesta dentro de un plazo determinado.}
  
  Ocimatic funciona de forma muy sencilla.
  \textbf{
  Los usuarios puede configurar el plazo $D$ en días que están dispuestos a
  esperar por una respuesta.
  Cada vez que un correo es enviado, Ocimatic configura un evento para que se
  ejecute luego de $D$ días.
  Cuando el evento es ejecutado transcurridos los $D$ días, Ocimatic verifica
  si el correo ya recibió una respuesta y notifica al usuario en caso de no
  haber recibido una.}
  Por ejemplo, si un usuario tiene configurado el plazo en 2 días y envía un
  correo un lunes, Ocimatic deberá verificar el miércoles si el correo ya recibió
  una respuesta.

  \textbf{Muchos de los usuarios han comenzado a quejarse porque desearían que Ocimatic
  solo considerara los días hábiles dentro del plazo.}
  Por ejemplo, si un usuario tiene configurado el plazo en 2 días y envía un
  correo el viernes, no espera ser notificado el domingo, sino el martes.
  Por otro lado, si envía un correo el sábado o el domingo, sabe que no lo
  leerán durante el fin de semana y es lo mismo que haberlo enviado el lunes,
  por lo tanto esperaría ser notificado el miércoles.

  Tomando en cuenta las quejas de sus usuarios los desarrolladores quieren
  modificar Ocimatic para que solo considere los días hábiles.
  Para esto, \textbf{necesitan saber la cantidad total de días, considerando los fines
  de semana, que Ocimatic debe esperar dada la cantidad de días hábiles $D$ que
  el usuario configuró.}
\end{problemDescription}

\begin{inputDescription}
  Un caso de prueba corresponde al envío de un correo y está descrito en una
  línea con dos enteros $D$ y $W$.
  $D$ corresponde a la cantidad de días hábiles que el usuario está dispuesto a
  esperar.
  $W$ es un entero entre 1 y 7 que corresponde al día de la semana en que el
  correo fue enviado (lunes=1, martes=2, miércoles=3, jueves=4, viernes=5,
  sábado=6 y domingo=7).
\end{inputDescription}

\begin{outputDescription}
  La salida debe contener un único entero correspondiente a la cantidad total de
  días que Ocimatic debe esperar antes de verificar si el correo recibió
  respuesta.
\end{outputDescription}

\newpage
\begin{scoreDescription}
  \score{5} Se probarán varios casos donde $W=1$ y $0 < D \leq 4$.

  Soluci\'on:

  Lo primero que debemos considerar, es que siempre es Lunes en esta subtarea.
  El máximo de días es 4, por lo que a lo más, se esperarán cuatro días. Eso
  significa que no tendremos que contar días del fin de semana. Por lo tanto, los
  días totales de espera son la misma cantidad que $D$.

  \score{10} Se probarán varios casos donde $1 \leq W \leq 5$ y $0 < D \leq 5$.

  Soluci\'on:

  En esta subtarea, ya no sirve la soluci\'on anterior, porque hay m\'ultiples casos
  en los que se deben contar d\'ias del fin de semana para el total. Por ejemplo, si 
  $D = 4$ y $W = 2$ (Martes), el cuarto d\'ia es S\'abado y Ocimatic esperar\'a hasta
  el Lunes para mandar el aviso, por lo que son 6 d\'ias de espera.

  Una cosa importante que debemos hacer primero es disminuir $W$ en uno para que
  la cuenta comience desde $0$ en vez de $1$.

  Luego, considerando los l\'imites de $D$ y $W$, s\'olo pasaremos por 1 fin 
  de semana. Entonces, tenemos dos opciones:
  \begin{enumerate}
    \item Si los valores de entrada no pasan por el fin de semana, retornamos $D$
    \item Si los valores de entrada pasan por el fin se semana, retornamos $D+2$
  \end{enumerate}

  Para saber si pasa por el fin de semana, la suma de $D + W$ debe superar el valor $4$.
  Esta suma supera $4$ siempre que la espera se pasa del Viernes.

  \score{18} Se probarán varios casos donde $0 < D \leq 10$.
  
  Soluci\'on

  Ahora, han aumentado los l\'imites de $D$ y adem\'as $W$ ya no tiene restricci\'on,
  o sea que puede tomar un valor de $1$ a $7$. La soluci\'on anterior ya no es suficiente,
  porque como $D$ puede llegar hasta $10$, podemos llegar a contar m\'as de un fin de 
  semana. Tambi\'en, debemos considerar qu\'e pasa si el correo se env\'ia un S\'abado
  o Domingo.

  Al igual que antes, debemos disminuir $W$ en uno para que la cuenta comience desde $0$
  en vez de $1$.

  El primer caso: si $W$ es un d\'ia de semana ($0 \geq W \geq 4$). Si la suma de
  d\'ias h\'abiles $D + W$ supera $4$, debemos a\~nadir $2$ d\'ias extras, como en el 
  \'item anterior. Si $D + W$ superan $9$, eso significa que ahora se deben considerar $2$
  fines de semana al total de d\'ias de espera $D$, por lo que se a\~naden $4$ d\'ias a
  la suma.
  
  El segundo caso: si $W$ es S\'abado o Domingo. En este caso, no vamos a considerar
  $W$ para cu\'antos d\'ias de fin debemos a\~nadir. Lo vamos a considerar s\'olo para
  la suma total de d\'ias de espera. Si el primer d\'ia era S\'abado, 
  debemos incorporar a los d\'ias de espera tanto S\'abado como Domingo. Si el primer 
  d\'ia era Domingo, debemos incorporar a la espera s\'olo el Domingo.
  
  Para representar que $W$ no se debe considerar para determinar cu\'antos fines de
  semana debemos contar, cambiamos su valor a cero despu\'es del paso del p\'arrafo
  anterior.
  
  Por \'ultimo, hacemos algo similar al primer caso: si $D$ supera $4$, debemos a\~nadir
  $2$ d\'ias extra, si supera $9$, debemos aumentar en $4$.

  \score{28} Se probarán varios casos donde $0 < D \leq 10^5$

  Soluci\'on

  Ahora que $D$ tiene l\'imites muy grandes, debemos pensar una soluci\'on que sea 
  general. Las que hemos hecho hasta el momento, s\'olo manejan casos particulares.
  De todas formas, podemos observar de ellas que hay un patr\'on que podemos seguir.

  Para cada soluci\'on usamos $D$ o $D + W$ para contar el n\'umero de fines de semana
  que debemos a\~nadir. Para saber cu\'antos fines de semana debemos sumar, debemos
  dividir $D$ o $D + W$ por $5$ para obtener el n\'umero de fines de semana. Cada vez
  que esta suma llega a un m\'ultiplo de cinco, quiere decir que pasamos por una semana
  completa, por eso es que dividimos por los cinco d\'ias de la semana de trabajo.

  Luego, esta cantidad se multiplica por $2$ para obtener el total de d\'ias que se
  deben a\~nadir al total.

  Teniendo el total de d\'ias de fin de semana que nos faltan, s\'olo se los a\~adimos a
  $D$, que son los d\'ias h\'abiles de espera.

  

  \score{39} Se probarán varios casos donde $0 < D \leq 10^{10}$ \\
  Soluci\'on
\end{scoreDescription}

\begin{sampleDescription}
\sampleIO{sample-3}
\sampleIO{sample-1}
\sampleIO{sample-2}
\sampleIO{sample-4}
\end{sampleDescription}

\end{document}
